\documentclass{article}
\usepackage[utf8]{inputenc}

\title{finanse}
\author{bartczukkuba }
\date{February 2020}

\begin{document}

\section{High level}

Information theory deals with properties of coding systems - these properties are defined in terms of sources and their probability distributions.

\section{Codes}

$S$ - source (domain), $T$ - target (alphabet). A code $\mathcal{C}$ is a function $S \to T^+$.

\subsection{Uniquely decodable codes}

\textbf{Theorem}

Let $C_0 = C$, $C_{n+1} = \{w \in T^+ | uw = v, u \in C_n, v \in C or \in C, v \in C_n \}$

Code is uniquely decodable if $\mathcal{C}_\infty \cap \mathcal{C} = \emptyset$

\subsection{Instantaneous Codes}


\end{document}
