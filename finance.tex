\documentclass{article}
\usepackage[utf8]{inputenc}

\title{finanse}
\author{bartczukkuba }
\date{February 2020}

\begin{document}

Procent składany - kapitalizacja

$$R_{t+1} = \frac{P_{t+1} - P_t}{P_t}$$

Total return - z uwzględnionymi dywidendami

Multi-period returns

$$R_{t+k, t} = \prod_{t \leq j \leq t+k-1}(1+R_{j+1, j}) - 1$$

\section{Measures of Risk and Reward}
Return on risk

Risk free rate

Cytat:

Although there is no perfectly riskless investment, the very short term US Treasury Bill (30 days or less) is typically used as a proxy for the risk free rate.

\subsection{Sharpe ratio}

$$S = \frac{R - R_{risk free}}{Volatility}$$

Where $Volatility$ is most likely some estimator of standard deviation

\subsection{Max Drawdown}

Sharpe ratio uses variance so it doesn't distinguish upward and downward volatility

Max drawdown - max loss between previous high to subsequent low

Calmar ratio

$$MDD = max_t max_{t' > t} P_t - P_{t'}$$

\section{Extreme risk estimates}

Sharpe ratio et c is useful when the outcomes follow Gaussian distribution. They might fail if the distribution in question has fat tails.

Var (value at risk)

\maketitle

\section{Introduction}

\end{document}
